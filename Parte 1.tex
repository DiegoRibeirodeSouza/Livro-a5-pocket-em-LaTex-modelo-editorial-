\part{Noções Introdutórias}



%%%%%%%%%%%%%%%%%%%%%%%%%%%%%%%%%%%%%%%%
%%%%%%%%%%%%%%%%%%%%%%%%%%%%%%%%%%%%%%%%

\chapter{Valores Sociais modernos}




Exemplo: Esse mesmo homem moderno,``jactando-se de ser civilizado e educado, não tem conseguido evitar as guerras que promove, utilizando-se dos notáveis descobrimentos que canaliza para o crime e a destruição, vitimado pela mesquinhez e pelos conflitos internos que o esmagam dolorosamente.''\footcite[p. 15]{embuscadaverdade}

\lipsum[1-5]



\section{Questões acerca disso e aquilo...}

Exemplo te texto: ... noção acerca dos limites do próprio conhecimento é relevante porque assim o homem pode ocupar-se com aquilo que potencialmente pode e deve conhecer.\footcite[p. 30]{locke1999ensaio}

O indivíduo, desse modo, poupa-se de reflexões fúteis à medida que entende que nem todo o conhecimento lhe é permitido saber.

De fato é laborioso distinguir o que é ou não campo de análise que valha a pena o ser humano dedicar seu tempo e esforço. Todavia,

\begin{quote}
	Passando o princípio inteligente por diversos patamares do processo
	da evolução, fixa todas as experiências que lhe constituem patrimônio
	de crescimento mental e moral, atravessando os períodos mais difíceis
	e laboriosos da fase inicial, para alcançar os níveis de lucidez que
	o capacitam à compreensão e vivência dos Soberanos Códigos que
	regem o Cosmo.\footcite[p.22]{JoannaDeAngelis2013Vida}
\end{quote}

Eis o motivo da dificuldade no que tange ao saber: é preciso, pois, muito tempo, dedicação e paciência para compreendermos cada vez mais aquilo que nos é necessário, de sorte que é absolutamente inútil estarmos cientes de tudo quanto se passa no globo terrestre, sob pena de nos inserirmos desnecessariamente a par de informações que em realidade nos são mais inúteis que útei

\lipsum[1-5]

% TODO: Configurar tamanho do capítulo

\subsection{A contemporaneidade}



\lipsum[1-5]

% TODO: Configurar tamanho do capítulo

\chapter{Crítica Aos Sistemas Políticos}

Os sistemas filosóficos, desde há muito apresentam teorias demasiadamente difíceis de serem digeridas pelo ser humano comum. A formulação de conceitos e teses demasiadamente variados, complexas, sem uniformidade fizeram antes um desfavor à filosofia, pois se multiplicou o desinteresse pela área.

Se tornaram, principalmente os eruditos desta área, semelhante

\begin{quote}
	as galinhas exaustas. Eles não constituem verdadeiramente nenhuma natureza “harmônica”: só conseguem cacarejar mais do que nunca porque põem ovos mais freqüentemente[sic]: é certo que os ovos também foram se tornando cada vez menores'' (por mais que os livros tenham se tornado cada vez mais grossos)\footcite[p. 64]{Nietzsche2003Segunda}
\end{quote}

E, em que pese o louvável esforço daqueles que buscaram contribuir para o avanço do saber humano na totalidade, grande foi o número daqueles que falaram \textit{coisa com coisa}, antes tornando a filosofia como sinônimo de fútil complexidade, ao invés de admirável doutrina em prol do saber....





%%%%%%%%%%%%%%%%%%%%%%%%%%%%%%%%%%%%%%%%
%%%%%%%%%%%%%%%%%%%%%%%%%%%%%%%%%%%%%%%%


% TODO: Configurar tamanho do capítulo

\part{Caminhos do Saber}

\chapter{Os diferentes passos }


\lipsum[1-5]

% TODO: Configurar tamanho do capítulo

\chapter{Limites do niilismo}


\lipsum[1-5]

Isto porque o saber, engessado que estava nas fileiras acadêmicas, cansado da incompletude de suas respostas para tudo, viu-se obrigado a sacudir-se, e renovar-se, embora o ceticismo ainda continue a liderar hodiernamente. Ocorre que

\begin{quote}
	Faltaram sempre à ciência oficial a independência e a liberdade; apartou-se do caminho, submetendo-se servilmente à autoridade da Igreja; em seguida, enfeudou-se às doutrinas materialistas do século XVIII e, em seguida, ao panteísmo germânico. Enfim, depois de quase um século, tornou-se o satélite do Positivismo, essa doutrina incompleta, que se desinteressa sistematicamente do maior problema que o espírito humano quer e deve resolver o da sua origem e de seu destino. Ela se limita a arrastar pelo mundo fórmulas secas e banais, semelhantes à "Vitória-aptera", que, desprovida de asas, se achava condenada a rastejar, sem poder elevar-se do solo.\footcite[p.112]{denis2008}
\end{quote}

Desse modo, temos receio de que aqueles que se julgam cientistas, não estejam, em realidade, tão presos na ortodoxia como os limitados sistemas de crenças que tão difusamente têm se alastrado no mundo desde os primórdios da humanidade.

\begin{quote}
	Enquanto os homens destas gerações, submetidos à disciplina da Igreja ou da Universidade, não tiverem desaparecido, apenas se poderá esboçar a obra de redenção do espírito humano. A Igreja com suas confissões e a Universidade com seus exames quebrantaram a elasticidade da Alma e oprimiram os surtos do pensamento.\footcite[p. 114]{denis2008}
\end{quote}

Não sendo assim, permaneceremos retrógrados, antiquados, ao passo que milhares de pssoas deambulam sem gosto pela vida após a completa ausência de sentido para suas vidas e o porque de tantos desafios na jornada terrestre.

\lipsum[1-5]



\section{Inovações nas Pesquisas Modernas Sobre Fenômenos Paranormais }

\lipsum[1-5]


\subsection{Inovações nas Pesquisas Modernas Sobre Fenômenos Paranormais }

\lipsum[1-5]

\subsubsection{Inovações nas Pesquisas Modernas Sobre Fenômenos Paranormais }


\lipsum[1-5]

% TODO: Adicionar texto
%%%%%%%%%%%%%%%%%%%%%%%%%%%%%%%%%%%%%%%%
%%%%%%%%%%%%%%%%%%%%%%%%%%%%%%%%%%%%%%%%


\chapter{Outro capítulo}

\lipsum[1-5]



% TODO: Configurar tamanho do capítulo




\chapter{Crises Chapter}


\lipsum[1-5]

%%%%%%%%%%%%%%%%%%%%%%%%%%%%%%%%%%%%%%%%
%%%%%%%%%%%%%%%%%%%%%%%%%%%%%%%%%%%%%%%%




\part{Lipsum part}

\chapter{Além das Atuais Limitações}

\lipsum[1-5]

% TODO: Configurar tamanho do capítulo


\chapter{Lipsun Chapter}



\lipsum[1-5]




\chapter{End Chapter }



\lipsum[1-5]
 


%%%%%%%%%%%%%%%%%%%%%%%%%%%%%%%%%%%%%%%%
%%%%%%%%%%%%%%%%%%%%%%%%%%%%%%%%%%%%%%%%


%%%%%%%%%%%%%%%%%%%%%%%%%%%%%%%%%%%%%%%%

