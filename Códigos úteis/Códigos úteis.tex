


_________Deixa a \part sem número no sumário_______________



%%%%%% Definição da \part{title} 
\makeatletter
% Modifica como as partes são adicionadas ao sumário
\renewcommand\part{%
	\if@openright\cleardoublepage\else\clearpage\fi
	\thispagestyle{plain}%
	\if@twocolumn\onecolumn\@tempswatrue\else\@tempswafalse\fi
	\null\vfil
	\secdef\@part\@spart}

\def\@part[#1]#2{%
	\ifnum \c@secnumdepth >-2\relax
	\refstepcounter{part}%
	\addcontentsline{toc}{part}{#1} % Adiciona apenas o título
	\else
	\addcontentsline{toc}{part}{#1}%
	\fi
	\markboth{}{}%
	{\centering
		\interlinepenalty \@M
		\normalfont
		\ifnum \c@secnumdepth >-2\relax
		\huge\bfseries \partname\nobreakspace\thepart
		\par
		\vskip 20\p@
		\fi
		\Huge \bfseries #2\par}%
	\@endpart}
\makeatother







_________Deixa a \part sem número no sumário e corpo do texto_______________

%%%%%% Definição da \part{title} 
% Configuração do sumário (remove número das partes)
\setlength{\cftpartnumwidth}{0pt}
\renewcommand{\cftpartpresnum}{}
\renewcommand{\cftpartaftersnum}{}

% Remove "Parte I" do corpo do texto
\makeatletter
\renewcommand{\@part}[2][?]{%
	\ifnum \c@secnumdepth >-2\relax
	\refstepcounter{part}%
	\addcontentsline{toc}{part}{#2}%
	\else
	\addcontentsline{toc}{part}{#2}%
	\fi
	\markboth{}{}%
	{\centering
		\interlinepenalty \@M
		\normalfont
		\Huge \bfseries #2\par}%
	\@endpart
}
\makeatother













_________codigo a ser avaliado (espaçamentos úteis)_______________


% Tente esta abordagem mais direta para remover o número da parte
\renewcommand{\cftpartpresnum}{}          % Remove o que vem ANTES do número (ex: "Parte")
\renewcommand{\cftpartfont}{\large\bfseries} % Define a fonte do título da parte
\renewcommand{\cftpartpagefont}{\bfseries}   % Define a fonte do número da página
\renewcommand{\cftpartleader}{\cftdotfill{\cftdotsep}} % Linha pontilhada
\renewcommand{\cftpartafterpnum}{\par\addvspace{10pt}} % Espaço após a entrada

% Comando principal: Redefine o formato do número para não exibir nada
\makeatletter
\renewcommand{\@cftmakepartnum}[1]{}
\makeatother









