% =========================================================================
% CLASSE DO DOCUMENTO E PACOTES BÁSICOS
% =========================================================================
\documentclass[12pt,openright]{book}  % Classe para livros, com capítulos começando sempre em páginas ímpares

\usepackage{layouts}         % Para inspecionar margens e layout visual
\usepackage{lipsum}          % Gera texto fictício (Lorem Ipsum) para testes
\usepackage{etoolbox}        % Ferramentas de programação condicional em LaTeX
\usepackage{setspace}        % Controle de espaçamento entre linhas
\usepackage{epigraph}        % Pacote para adicionar epígrafes
\usepackage{ragged2e}    
\usepackage{tabularx}   
\usepackage{microtype}
\usepackage{hyperref}
\hypersetup{hidelinks}
\usepackage{todonotes}





\usepackage{fontspec}
\usepackage{unicode-math}



\setmainfont{Sabon MT Std}[
UprightFont = sabon-mt-std.otf,
ItalicFont = sabon-mt-std-italic.otf,
BoldFont = sabon-mt-std-semibold.otf,
BoldItalicFont = sabon-mt-std-semibold-italic.otf
]
\setsansfont{Urbanist-SemiBold}
\setmonofont{Source Code Pro}
\setmathfont{Latin Modern Math}

% =========================================================================
% FORMATAÇÃO DE TÍTULOS
% =========================================================================
\usepackage{titlesec}        % Permite personalizar títulos de capítulos e seções

% Numeração em algarismos romanos hierárquicos
\renewcommand{\thechapter}{\Roman{chapter}}
\renewcommand{\thesection}{\thechapter.\Roman{section}}
\renewcommand{\thesubsection}{\thesection.\Roman{subsection}}
\renewcommand{\thesubsubsection}{\thesubsection.\Roman{subsubsection}}

% ----------------------------------------
% Formatação dos títulos dos capítulos
% ----------------------------------------
\titleformat{\chapter}[block]
{\normalfont\large\bfseries\raggedleft}
{\thechapter\\}
{0pt}
{}
\titlespacing*{\chapter}{0pt}{0pt}{1em}

% Redefine visual do cabeçalho de capítulo
\makeatletter
\renewcommand{\@makechapterhead}[1]{%
	\vspace*{60pt}
	{\centering\normalfont\sffamily\large\bfseries
		\thechapter\par}
	\vspace{0.6em}
	{\centering\hrule height 0.5pt width 0.2\textwidth}
	\vspace{1em}
	{\raggedleft\normalfont\sffamily\large\bfseries
		#1\par\nobreak}
	\vspace{2.5em}
}



% Formatação das seções e subseções com fonte sans-serif
\titleformat{\section}
{\normalfont\sffamily\normalsize\bfseries}
{\thesection}{0.5em}{}
\titlespacing*{\section}{0pt}{1.5ex plus .1ex minus .2ex}{1ex plus .1ex}

\titleformat{\subsection}
{\normalfont\sffamily\normalsize\bfseries}
{\thesubsection}{0.5em}{}
\titlespacing*{\subsection}{0pt}{1.5ex plus .1ex minus .2ex}{1ex plus .1ex}

\titleformat{\subsubsection}
{\normalfont\sffamily\normalsize\bfseries}
{\thesubsubsection}{0.5em}{}
\titlespacing*{\subsubsection}{0pt}{1.5ex plus .1ex minus .2ex}{1ex plus .1ex}

% =========================================================================
% CONFIGURAÇÃO DO SUMÁRIO (TABLE OF CONTENTS)
% =========================================================================
\usepackage{tocloft} % Personalização do sumário (ToC)

% Título do sumário
\renewcommand{\cfttoctitlefont}{\hfill\normalfont\large\sffamily\bfseries}
\renewcommand{\cftaftertoctitle}{\hfill}

% Capítulos no sumário
\renewcommand{\cftchapfont}{\normalfont\normalsize\rmfamily\raggedleft}
\renewcommand{\cftchappagefont}{\normalfont\normalsize\rmfamily}
\renewcommand{\cftchapleader}{\cftdotfill{\cftdotsep}}
\setlength{\cftchapindent}{0em}
\setlength{\cftchapnumwidth}{2.9em}
\renewcommand{\cftchapaftersnum}{\hspace{0.5em}} % Espaço após o número do capítulo

% Seções
\renewcommand{\cftsecfont}{\normalfont\normalsize\rmfamily}
\renewcommand{\cftsecpagefont}{\normalfont\normalsize\rmfamily}
\renewcommand{\cftsecleader}{\cftdotfill{\cftdotsep}}
\setlength{\cftsecindent}{0.9em}
\setlength{\cftsecnumwidth}{3.5em}
\renewcommand{\cftsecaftersnum}{\hspace{0.9em}} % Espaço após o número da seção

% Subseções
\renewcommand{\cftsubsecfont}{\normalfont\normalsize\rmfamily}
\renewcommand{\cftsubsecpagefont}{\normalfont\normalsize\rmfamily}
\renewcommand{\cftsubsecleader}{\cftdotfill{\cftdotsep}}
\setlength{\cftsubsecindent}{1.2em}
\setlength{\cftsubsecnumwidth}{2.5em}
\renewcommand{\cftsubsecaftersnum}{\hspace{0.7em}} % Espaço após o número da subseção

% Subsubseções
\renewcommand{\cftsubsubsecfont}{\normalfont\normalsize\rmfamily}
\renewcommand{\cftsubsubsecpagefont}{\normalfont\normalsize\rmfamily}
\renewcommand{\cftsubsubsecleader}{\cftdotfill{\cftdotsep}}
\setlength{\cftsubsubsecindent}{5.4em}
\setlength{\cftsubsubsecnumwidth}{2.0em}
\renewcommand{\cftsubsubsecaftersnum}{\hspace{0.7em}} % Espaço após o número da subsubseção

% Partes
\renewcommand{\cftpartfont}{\normalfont\normalsize\rmfamily\bfseries}
\renewcommand{\cftpartpagefont}{\normalfont\normalsize\rmfamily}
\renewcommand{\cftpartleader}{\cftdotfill{\cftdotsep}}
\setlength{\cftpartindent}{0em}
\setlength{\cftpartnumwidth}{2.5em}
\renewcommand{\cftpartaftersnum}{\hspace{0.5em}}

% Espaçamento vertical entre entradas no sumário
\setlength{\cftbeforechapskip}{5pt}
\setlength{\cftbeforesecskip}{5pt}
\setlength{\cftbeforesubsecskip}{5pt}
\setlength{\cftbeforepartskip}{1\baselineskip}  


% =========================================================================
% FORMATAÇÃO DE PARTES
% =========================================================================
\renewcommand\part{%
	\if@openright\cleardoublepage\else\clearpage\fi
	\thispagestyle{plain}%
	\if@twocolumn\onecolumn\@tempswatrue\else\@tempswafalse\fi
	\null\vfil
	\secdef\@part\@spart}

\makeatletter
\def\@part[#1]#2{%
	\ifnum \c@secnumdepth >-2\relax
	\refstepcounter{part}%
	\addcontentsline{toc}{part}{\protect\partname\ \thepart: #1}% <-- LINHA CORRIGIDA
	\else
	\addcontentsline{toc}{part}{#1}%
	\fi
	\markboth{}{}%
	\vspace*{60pt}
	{\raggedleft\normalfont\sffamily\large\bfseries % Alterado de \centering para \raggedleft
		\partname\nobreakspace\thepart\par}
	\vspace{0.6em}
	\vspace{1em}
	{\raggedleft\normalfont\sffamily\large\bfseries
		#2\par\nobreak}
	\vspace{2.5em}
	\@endpart}
\makeatother
% =========================================================================
% CONFIGURAÇÃO DAS CITAÇÕES LONGAS (BLOCO DE QUOTE)
% =========================================================================
\makeatletter
\renewenvironment{quote}{%
	\par
	\addvspace{0.58\baselineskip}%
	\small
	\setstretch{0.9}%
	\list{}{%
		\leftmargin=1.4cm
		\rightmargin=0cm
		\parsep=0pt
		\itemsep=0pt
		\topsep=0pt
		\partopsep=0pt
	}%
	\item\relax
}{%
	\endlist%
	\addvspace{0.4\baselineskip}%
	\par
}
\makeatother

% =========================================================================
% CONFIGURAÇÕES DA BIBLIOGRAFIA E CITAÇÕES
% =========================================================================
\usepackage{csquotes}  % Recomendado pelo biblatex para citações apropriadas



\usepackage[
style=verbose-ibid,  % Estilo com repetições ibid.
backend=biber,
citetracker=true,
ibidtracker=true
]{biblatex}
\addbibresource{referencias.bib}  % Base de dados bibliográficos


% Redefine a macro interna que exibe o título e subtítulo.


\DeclareFieldFormat{title}{\emph{#1}}
% Isso permite um controle mais granular.
\renewbibmacro*{title}{%
	\iffieldundef{title}
	{} % Faz nada se não houver título
	{%
		% Imprime o título principal (já configurado para itálico por \DeclareFieldFormat{title})
		\printfield{title}%
		% Se houver subtítulo, adiciona a pontuação e imprime em fonte normal.
		\iffieldundef{subtitle}
		{}
		{%
			\setunit{\subtitlepunct}% % Adiciona a pontuação entre título e subtítulo (geralmente dois pontos)
			\normalfont\printfield{subtitle}% % <--- Aqui o subtítulo é forçado a ser normal
		}%
		\newunit%
	}%
}



% Título da bibliografia e entrada no sumário
\defbibheading{bibliography}[\refname]{%
	{\raggedleft\normalfont\sffamily\large\bfseries
		#1\par\nobreak}
	\vspace{1em}
	\addcontentsline{toc}{chapter}{#1}
}

% --- Formatação dos nomes ---
\DeclareNameAlias{default}{family-given} % Formato "SOBRENOME, Nome"
\DeclareNameAlias{sortname}{family-given} % Para ordenação alfabética
\DeclareNameAlias{labelname}{family-given} % Para citações no texto

% --- Sobrenome em maiúsculas, nome normal ---

\renewcommand*{\mkbibnameprefix}[1]{#1} % Prefixos (ex: "da", "de")
\renewcommand*{\mkbibnamesuffix}[1]{#1} % Sufixos (ex: "Júnior")

% --- Estilo de exibição ---
\DeclareNameFormat{family-given}{%
	\nameparts{#1}%
	\usebibmacro{name:family-given}%
	{\MakeUppercase{\namepartfamily}} % Sobrenome MAIÚSCULO
	{\namepartgiven}                 % Nome normal
	{\namepartprefix}%
	{\namepartsuffix}%
	\usebibmacro{name:andothers}%
}

% Tradução do título da bibliografia
\DefineBibliographyStrings{brazil}{
	bibliography = {Referências},
}

% =========================================================================
% CONFIGURAÇÕES DE NOTAS DE RODAPÉ
% =========================================================================
\counterwithout{footnote}{chapter}  % Notas com numeração contínua entre capítulos
\usepackage[hang]{footmisc}         % Alinhamento pendente nas notas
\setlength{\footnotesep}{12pt}
\setlength{\footnotemargin}{0pt}

% Estética das notas
\makeatletter
\renewcommand\@makefntext[1]{%
	\noindent
	{\footnotesize\textsuperscript{\@thefnmark}}%
	\hspace{0.3em}#1%
}
\makeatother

% =========================================================================
% METADADOS DO LIVRO
% =========================================================================


% Definições reutilizáveis no documento
\newcommand{\authorname}{Diego Ribeiro de Souza}
\newcommand{\translatorname}{--}
\newcommand{\booktitle}{Filosofia e Espiritismo}
\newcommand{\subtitle}{Ensaios Sobre os Rumos das Civilizações}
\newcommand{\publisher}{KDP}
\newcommand{\editionyear}{2025}
\newcommand{\isbn}{123-456-789-0}

\usepackage{misc/options}  % Configurações opcionais adicionais do usuário

% --- Redefine o formato da epígrafe para justificar o texto ---
\makeatletter
\renewcommand{\@epitext}[1]{
	\begin{minipage}{\epigraphwidth}
		\justify % Justifica o texto
		#1
\end{minipage}}
\makeatother



% =========================================================================
% INÍCIO DO DOCUMENTO
% =========================================================================
\begin{document}
	
	% Páginas iniciais com numeração romana (índice, prefácio, etc.)
	\frontmatter
	\pagestyle{empty}

% =======================================================
% Capa personalizada (imagem PNG)
% =======================================================
\thispagestyle{empty}
\newgeometry{margin=0cm} % Remove as margens para a imagem preencher a página

\begin{figure}[p]
	\centering
	\includegraphics[width=\paperwidth, height=\paperheight, keepaspectratio=false]{/home/diego/Documentos/Livro a5 (pocket) em LaTex (modelo editorial)/frontmatter/Capa.png}
\end{figure}

\restoregeometry % Restaura as margens originais do documento
\cleardoublepage


% =======================================================
% PÁGINA DE TÍTULO (PADRÃO EDITORIAL REFINADO)
% =======================================================
\begin{titlepage}
	\thispagestyle{empty} % Remove o número de página desta folha
	\centering % Centraliza todo o conteúdo horizontalmente
	
	\vspace*{\stretch{2}} % Espaço flexível no topo
	
	% Título do Livro
	{\fontsize{30}{36}\bfseries\booktitle\par} % Título
	\vspace{12pt}
	
	% Subtítulo (se existir)
	\ifx\subtitle\undefined\else
	\if\relax\detokenize\expandafter{\subtitle}\relax\else
	{\large\normalfont\subtitle\par} % Subtítulo
	\fi
	\fi
	
	\vspace*{\stretch{40}} % Espaço flexível entre o título e o autor
	
	% Nome do Autor
	{\authorname\par}
	
	\vspace*{\stretch{6}} % Espaço flexível para empurrar a editora para baixo
	
	% Editora
	{\scshape\publisher\par}
	
	\vfill % Espaço flexível final
\end{titlepage}

% =======================================================
% FICHA CATALOGRÁFICA (COM RODAPÉ NA MESMA PÁGINA)
% =======================================================

\thispagestyle{empty}
\pagestyle{empty}

% Espaço do topo da página para a ficha
\vspace*{1.5cm}

\begin{center}
	% Este minipage grande agrupa toda a ficha e o rodapé
	\begin{minipage}{0.9\textwidth}
		\begin{center}
			{\small\scshape Ficha Catalográfica} \\[10pt]
			{\footnotesize(Conforme Decreto nº 4.591, de 17/12/2002)} \\[10pt]
		\end{center}
		
		\begin{center}
			\fbox{%
				\begin{minipage}{\textwidth}
					\centering
					\footnotesize % Tamanho da fonte para a ficha
					\begin{tabular}{@{}p{\dimexpr\textwidth-2\tabcolsep}@{}}
						\rule{\linewidth}{0.4pt} \\[6pt]
						
						SOUZA, Diego Ribeiro de \\[3pt]
						\hspace{8pt}Filosofia e espiritismo: ensaios sobre os rumos das civilizações / Diego Ribeiro de Souza. -- Ed.~\the\year. -- \publisher, \the\year. \\[3pt]
						\hspace{8pt}\pageref{LastPage}~p. ; 15~cm $\times$ 21~cm \\[6pt]
						
						\hspace{8pt}ISBN: \isbn \\[6pt]
						
						\hspace{8pt}1. Filosofia espírita. 2. Espiritualidade. 3. Metafísica. \\
						\hspace{8pt}4. Espiritismo. I. Título. II. Série. \\[9pt]
						
						CDU: 141.33 \\[6pt]
						\rule{\linewidth}{0.4pt}
					\end{tabular}
					\vspace{9pt}
					\begin{flushright}
						\footnotesize
						\begin{tabular}{r@{}}
							Catalogação na publicação: \\
							Bibliotecária Responsável: \textbf{[Nome Completo]} \\
							CRB-[XX]/[Número] \\
						\end{tabular}
					\end{flushright}
				\end{minipage}
			}
		\end{center}
		
		% Espaçamento entre a ficha e o rodapé
		\vspace*{1cm}
		
		\begin{center}
			\footnotesize % Tamanho da fonte para o rodapé
			\publisher\ -- [Endereço completo] -- Tel.: (XX) XXXX-XXXX \\
			E-mail: contato@editora.com -- www.editora.com.br
		\end{center}
	\end{minipage}
\end{center}

\vfill     % Página de título
	% =======================================================
% PÁGINA DE DIREITOS AUTORAIS (PADRÃO EDITORIAL REFINADO)
% =======================================================
\cleardoublepage % Garante que a página comece em uma página ímpar
\thispagestyle{empty} % Remove o número de página desta folha
\vspace*{\fill} % Espaço flexível no topo

\begin{center}
	\begin{minipage}{0.8\textwidth}
		\linespread{1.1}\selectfont % Ajusta o espaçamento entre linhas
		
		\footnotesize % Tamanho da fonte para a maioria do texto
		
		\textbf{Copyright \textcopyright\ \the\year\ \authorname} \\
		Todos os direitos reservados. \\
		Nenhuma parte deste livro pode ser reproduzida ou usada de forma alguma sem a permissão expressa do autor, exceto para o uso de citações em resenhas ou críticas. \\[1em]
		
		\ifx\isbn\undefined\else
		\textbf{ISBN:} \isbn \\[1em]
		\fi
		
		\textbf{Editora:} \publisher \\[0.5em]
		[Endereço da editora, se necessário] \\[0.5em]
		[Telefone e E-mail da editora] \\[0.5em]
		[Site da editora] \\[2em]
		
		% Inclusão do logo (garantir que a imagem exista)
		\includegraphics[width=1cm]{frontmatter/logo-black.png} % Ajustei a largura do logo para um tamanho comum
		
	\end{minipage}
\end{center}

\vspace*{\fill} % Espaço flexível na parte inferior
\cleardoublepage % Garante que a próxima página seja uma página ímpar % Página de direitos autorais
	\input{frontmatter/preface}       % Prefácio
	\input{frontmatter/tocpage}       % Sumário
	
	% Conteúdo principal com numeração normal
	\mainmatter
	\pagestyle{fancy}
	\setcounter{page}{13}             % Define o número da primeira página do conteúdo
	\part{Noções Introdutórias}



%%%%%%%%%%%%%%%%%%%%%%%%%%%%%%%%%%%%%%%%
%%%%%%%%%%%%%%%%%%%%%%%%%%%%%%%%%%%%%%%%

\chapter{Valores Sociais modernos}




Exemplo: Esse mesmo homem moderno,``jactando-se de ser civilizado e educado, não tem conseguido evitar as guerras que promove, utilizando-se dos notáveis descobrimentos que canaliza para o crime e a destruição, vitimado pela mesquinhez e pelos conflitos internos que o esmagam dolorosamente.''\footcite[p. 15]{embuscadaverdade}

\lipsum[1-5]



\section{Questões acerca disso e aquilo...}

Exemplo te texto: ... noção acerca dos limites do próprio conhecimento é relevante porque assim o homem pode ocupar-se com aquilo que potencialmente pode e deve conhecer.\footcite[p. 30]{locke1999ensaio}

O indivíduo, desse modo, poupa-se de reflexões fúteis à medida que entende que nem todo o conhecimento lhe é permitido saber.

De fato é laborioso distinguir o que é ou não campo de análise que valha a pena o ser humano dedicar seu tempo e esforço. Todavia,

\begin{quote}
	Passando o princípio inteligente por diversos patamares do processo
	da evolução, fixa todas as experiências que lhe constituem patrimônio
	de crescimento mental e moral, atravessando os períodos mais difíceis
	e laboriosos da fase inicial, para alcançar os níveis de lucidez que
	o capacitam à compreensão e vivência dos Soberanos Códigos que
	regem o Cosmo.\footcite[p.22]{JoannaDeAngelis2013Vida}
\end{quote}

Eis o motivo da dificuldade no que tange ao saber: é preciso, pois, muito tempo, dedicação e paciência para compreendermos cada vez mais aquilo que nos é necessário, de sorte que é absolutamente inútil estarmos cientes de tudo quanto se passa no globo terrestre, sob pena de nos inserirmos desnecessariamente a par de informações que em realidade nos são mais inúteis que útei

\lipsum[1-5]

% TODO: Configurar tamanho do capítulo

\subsection{A contemporaneidade}



\lipsum[1-5]

% TODO: Configurar tamanho do capítulo

\chapter{Crítica Aos Sistemas Políticos}

Os sistemas filosóficos, desde há muito apresentam teorias demasiadamente difíceis de serem digeridas pelo ser humano comum. A formulação de conceitos e teses demasiadamente variados, complexas, sem uniformidade fizeram antes um desfavor à filosofia, pois se multiplicou o desinteresse pela área.

Se tornaram, principalmente os eruditos desta área, semelhante

\begin{quote}
	as galinhas exaustas. Eles não constituem verdadeiramente nenhuma natureza “harmônica”: só conseguem cacarejar mais do que nunca porque põem ovos mais freqüentemente[sic]: é certo que os ovos também foram se tornando cada vez menores'' (por mais que os livros tenham se tornado cada vez mais grossos)\footcite[p. 64]{Nietzsche2003Segunda}
\end{quote}

E, em que pese o louvável esforço daqueles que buscaram contribuir para o avanço do saber humano na totalidade, grande foi o número daqueles que falaram \textit{coisa com coisa}, antes tornando a filosofia como sinônimo de fútil complexidade, ao invés de admirável doutrina em prol do saber....





%%%%%%%%%%%%%%%%%%%%%%%%%%%%%%%%%%%%%%%%
%%%%%%%%%%%%%%%%%%%%%%%%%%%%%%%%%%%%%%%%


% TODO: Configurar tamanho do capítulo

\part{Caminhos do Saber}

\chapter{Os diferentes passos }


\lipsum[1-5]

% TODO: Configurar tamanho do capítulo

\chapter{Limites do niilismo}


\lipsum[1-5]

Isto porque o saber, engessado que estava nas fileiras acadêmicas, cansado da incompletude de suas respostas para tudo, viu-se obrigado a sacudir-se, e renovar-se, embora o ceticismo ainda continue a liderar hodiernamente. Ocorre que

\begin{quote}
	Faltaram sempre à ciência oficial a independência e a liberdade; apartou-se do caminho, submetendo-se servilmente à autoridade da Igreja; em seguida, enfeudou-se às doutrinas materialistas do século XVIII e, em seguida, ao panteísmo germânico. Enfim, depois de quase um século, tornou-se o satélite do Positivismo, essa doutrina incompleta, que se desinteressa sistematicamente do maior problema que o espírito humano quer e deve resolver o da sua origem e de seu destino. Ela se limita a arrastar pelo mundo fórmulas secas e banais, semelhantes à "Vitória-aptera", que, desprovida de asas, se achava condenada a rastejar, sem poder elevar-se do solo.\footcite[p.112]{denis2008}
\end{quote}

Desse modo, temos receio de que aqueles que se julgam cientistas, não estejam, em realidade, tão presos na ortodoxia como os limitados sistemas de crenças que tão difusamente têm se alastrado no mundo desde os primórdios da humanidade.

\begin{quote}
	Enquanto os homens destas gerações, submetidos à disciplina da Igreja ou da Universidade, não tiverem desaparecido, apenas se poderá esboçar a obra de redenção do espírito humano. A Igreja com suas confissões e a Universidade com seus exames quebrantaram a elasticidade da Alma e oprimiram os surtos do pensamento.\footcite[p. 114]{denis2008}
\end{quote}

Não sendo assim, permaneceremos retrógrados, antiquados, ao passo que milhares de pssoas deambulam sem gosto pela vida após a completa ausência de sentido para suas vidas e o porque de tantos desafios na jornada terrestre.

\lipsum[1-5]



\section{Inovações nas Pesquisas Modernas Sobre Fenômenos Paranormais }

\lipsum[1-5]


\subsection{Inovações nas Pesquisas Modernas Sobre Fenômenos Paranormais }

\lipsum[1-5]

\subsubsection{Inovações nas Pesquisas Modernas Sobre Fenômenos Paranormais }


\lipsum[1-5]

% TODO: Adicionar texto
%%%%%%%%%%%%%%%%%%%%%%%%%%%%%%%%%%%%%%%%
%%%%%%%%%%%%%%%%%%%%%%%%%%%%%%%%%%%%%%%%


\chapter{Outro capítulo}

\lipsum[1-5]



% TODO: Configurar tamanho do capítulo




\chapter{Crises Chapter}


\lipsum[1-5]

%%%%%%%%%%%%%%%%%%%%%%%%%%%%%%%%%%%%%%%%
%%%%%%%%%%%%%%%%%%%%%%%%%%%%%%%%%%%%%%%%




\part{Lipsum part}

\chapter{Além das Atuais Limitações}

\lipsum[1-5]

% TODO: Configurar tamanho do capítulo


\chapter{Lipsun Chapter}



\lipsum[1-5]




\chapter{End Chapter }



\lipsum[1-5]
 


%%%%%%%%%%%%%%%%%%%%%%%%%%%%%%%%%%%%%%%%
%%%%%%%%%%%%%%%%%%%%%%%%%%%%%%%%%%%%%%%%


%%%%%%%%%%%%%%%%%%%%%%%%%%%%%%%%%%%%%%%%

                   % Insere conteúdo da Parte 1
	
	% \backmatter opcional, se desejar separar apêndices ou posfácio
	
% =========================================================================
% BIBLIOGRAFIA E FINALIZAÇÃO (VERSÃO FINAL AJUSTADA)
% =========================================================================

\backmatter

% Formatação manual do capítulo de referências
\clearpage
\thispagestyle{plain}

% Título alinhado à direita, sem maiúsculas, com mesmo estilo dos capítulos
\vspace*{60pt}%
{\raggedleft\normalfont\sffamily\large\bfseries
	Referências\par}
\vspace{0.6em}%
{\raggedleft\hrule height 0.5pt width 0.2\textwidth\par}
\vspace{1em}%

% Adiciona ao sumário
\addcontentsline{toc}{chapter}{Referências}

% Imprime as referências
\printbibliography[heading=none]

% 2. PÁGINA EM BRANCO APÓS REFERÊNCIAS (SEM NÚMERO)
\clearpage % Não usa \cleardoublepage para não forçar página ímpar
\pagestyle{empty} % Remove TODOS os cabeçalhos/rodapés
\hbox{} % Conteúdo vazio
\newpage % Força nova página


% Contracapa (opcional)
\cleardoublepage
\pagestyle{fancy}

\pagestyle{empty}  % Disable headers and footers on this page

\cleardoublepage    % Ensure the following content starts on a right-hand page in double-sided layouts



% =================
% Back cover design
% =================
\begin{center}
	\begin{tikzpicture}[remember picture, overlay]
		% Fundo sóbrio
		\fill[black] (current page.south west) rectangle (current page.north east);
		
		% Elemento central minimalista
	
		
		% Área de texto (3 parágrafos)
		\node[white!85, text width=0.65\paperwidth, align=justify, font=\small, 
		anchor=center] at ([yshift=0.5cm] current page.center) {
			\textsc{Este livro aborda a interseção entre filosofia e espiritualidade, explorando como os grandes pensadores da história contemplaram os mistérios da existência.}
			
			\vspace{0.3cm}
			\textsc{A obra conduz o leitor por uma jornada reflexiva, desde os pré-socráticos até os pensadores contemporâneos, estabelecendo diálogos com as tradições espirituais orientais e ocidentais.}
			
			\vspace{0.3cm}
			\textsc{Com linguagem acessível mas profunda, o autor oferece novas perspectivas sobre a natureza da consciência e nosso lugar no cosmos.}
		};
		
		% Elementos decorativos inferiores
		\foreach \x in {0.5,0.7,0.9} {
			\draw[white!15, line width=0.2pt] 
			([xshift=-\x*4cm, yshift=-4cm] current page.center) -- 
			([xshift=\x*4cm, yshift=-4cm] current page.center);
		}
		
		% Informações editoriais (parte inferior)
		\node[white!60, font=\tiny, anchor=south west] 
		at ([xshift=1.5cm, yshift=1cm] current page.south west) 
		{ISBN 978-XX-XXXXX-XX-X};
		
		\node[white!60, font=\tiny, anchor=south east] 
		at ([xshift=-1.5cm, yshift=1cm] current page.south east) 
		{\publisher};
	\end{tikzpicture}
\end{center}
\label{LastPage}


	
\end{document}
